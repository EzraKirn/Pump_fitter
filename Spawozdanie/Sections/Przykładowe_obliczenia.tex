\section{Przykładowe obliczenia}

\begin{flushleft}
    Wyliczenie mocy hydraulicznej
    \begin{equation}
        P_h=Q\cdot H\cdot \rho_{H_2O}\cdot g 
    \end{equation}
    \centering
    \(P_h=0.1\FLOWRATEh\cdot 101.3m \cdot 997 \DENSITY \cdot 9.81 \ACCELERATION = 27.5498 \frac{kg\cdot m^2}{s^3 \cdot 3600}\)
\end{flushleft} 

\begin{flushleft}
    Wyliczenie mocy sprawności
    \begin{equation}
        \eta = \frac{P}{P_h} 
    \end{equation}
    \centering
    \(\eta = \frac{22422.2 W}{9362.88 W}=0.417571\)
\end{flushleft} 

\begin{flushleft}
    Wyliczenie mocy na metr sześcienny przepływu
    \begin{equation}
        e=\frac{P}{Q} 
    \end{equation}
    \centering
    \(e=\frac{\frac{22195.6 W}{1000}}{32.8001\FLOWRATEh} =0.676693\frac{kW\cdot H}{m^3}\)
\end{flushleft} 

\begin{flushleft}
    Wyliczenie współczynnika R
    \begin{equation}
        R=\frac{H_d-H_g}{Q^2}
    \end{equation}
    \centering
    \(R=\frac{91 m - 2 m}{147^2\FLOWRATEh}=0.004118654\)
\end{flushleft}

\begin{flushleft}
    Wyliczenie równania pompy
    \begin{equation}
        dHu(Q)=H_g+RQ^2
    \end{equation}
    \centering
    \(dHu(Q)=2+ 0.004118654 Q^2\)
\end{flushleft} 

\begin{flushleft}
    Wyliczenie dokładności doboru
    \begin{equation}
        w_1=\frac{10}{\sqrt{\left(\lvert Q_{p.pracy}-Q_d\rvert\right)^2+\left(\lvert H_{p.pracy}-H_{d_d}\rvert\right)^2}}
    \end{equation}
\end{flushleft} 

\begin{flushleft}
    Wyliczenie sprawności w założonym punkcie pracy
    \begin{equation}
        w_2=\frac{Q_d\cdot H\cdot \rho_{H_2O}\cdot g}{P(Q_d)}
    \end{equation}
\end{flushleft} 

\begin{flushleft}
    Wyliczenie mocy na metr sześcienny w założonym punkcie pracy
    \begin{equation}
        w_3=\frac{P(Q_d)}{Q_d}
    \end{equation}
\end{flushleft} 

\begin{flushleft}
    Ocena wyboru
    \begin{equation}
        \sum_{n = 1}^{\infty}  w_n\cdot i_n
    \end{equation}
\end{flushleft} 

    

