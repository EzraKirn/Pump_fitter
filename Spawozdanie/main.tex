\documentclass[
12pt, % Main document font size
a4paper, % Paper type, use 'letterpaper' for US Letter paper
oneside, % One page layout (no page indentation)
%twoside, % Two page layout (page indentation for binding and different headers)
]{article}

\include{../../structure.tex}
\newcommand{\titleterma}[5]{
    % 1.numer sprawka
    % 2.temat
    % 3.nr grupy (N00-x0)
    % 4.data wykonania(dzien.miesiac)
    % 5.data oddania
    \begin{center}
        \fontsize{18}{1}\selectfont{ WYDZIAŁ MECHANICZNO-ENERGETYCZNY\\\vspace{10pt}
        POLITECHNIKI WROCŁAWSKIEJ }\\ \vspace{10pt}
        \textbf{\fontsize{26}{30}\selectfont LABORATORIUM TERMODYNAMIKI\\
        INSTYTUTU TECHNIKI CIEPLNEJ I MECHANIKI PŁYNÓW\\}
        \fontsize{16}{30}\selectfont Sprawozdanie nr. #1\\
        \textbf{Temat:}#2\\
        
    \end{center}
    \begin{flushleft}
        Grupa nr.#3 \\
        Skład podgrupy:
        \begin{enumerate}
            \item Grzegorz Wyborski 260906
            \item Anna Ziobrowska 255583
            \item Kacper Kasprzak 
        \end{enumerate}
        Termin zajęć: Czwartek, 18:55 20:30 \\
        Prowadzący: Mgr inż. Daria Krasota\\
        Data wykonania ćwiczenia: #4.2022 r.\\ 
        Data oddania sprawozdania: #5.2022 r.\\
        \vspace{20pt}
        \underline{Sprawozdanie powinno zawierać:}\\
        \begin{enumerate}
            \itemsep2pt{
            \item Podstawy teoretyczne
            \item Schemat układu pomiarowego
            \item Wykaz przyrządów pomiarowych
            \item Tabele pomiarowo-wynikowe
            \item Przykłady obliczeń
            \item Wykresy podanych zależności
            \item Uwagi, spostrzeżenia i wnioski
            \item Podpisany protokół z badań}
        \end{enumerate}
    \end{flushleft}
    \pagebreak
}

\newcommand{\titleelektra}[5]{
    \begin{center}
        \fontsize{18}{1}\selectfont{ Wydział Mechaniczno-Energetyczny\\
            Politechnika Wrocławska }\\ \vspace{10pt}
        \textbf{\fontsize{26}{30}\selectfont Podstawy Elektroniki\\ LABORATORIUM\\}
        \fontsize{16}{30}\selectfont Sprawozdanie nr. #1\\
        \textbf{Temat:}#2\\
        
    \end{center}
    \begin{flushleft}
        Grupa nr.#3 \\
        Skład podgrupy:
        \begin{enumerate}
            \item Grzegorz Wyborski 260906
            \item Ernest Kauch 260878
            \item Wojciech Mazur 252222
        \end{enumerate}
        Termin zajęć: Czwartek, 11:15-13.00 \\
        Prowadzący: Dr inż. Zbigniew Rogala\\
        Data wykonania ćwiczenia: #4.2022 r.\\ 
        Data oddania sprawozdania: #5.2022 r.\\
        \vspace{20pt}
        \underline{Sprawozdanie powinno zawierać:}\\
        \begin{enumerate}
            \itemsep 2pt{
            \item Podstawy teoretyczne
            \item Schemat układu pomiarowego
            \item Wykaz przyrządów pomiarowych
            \item Tabele pomiarowo-wynikowe
            \item Przykłady obliczeń
            \item Wykresy podanych zależności
            \item Uwagi, spostrzeżenia i wnioski
            \item Podpisany protokół z badań}
        \end{enumerate}
    \end{flushleft}
    \pagebreak
}

\newcommand{\titlebiomasa}[5]{
    \begin{center}
        \fontsize{18}{1}\selectfont{ Wydział Mechaniczno-Energetyczny\\
            Politechnika Wrocławska }\\ \vspace{10pt}
        \textbf{\fontsize{26}{30}\selectfont TECHNOLOGIE WYKORZYSTANIA BIOMASY\\ LABORATORIUM\\}
        \fontsize{16}{30}\selectfont Sprawozdanie nr. #1\\
        \textbf{Temat:}#2\\
    \end{center}
    \begin{flushleft}
        Grupa nr.#3 \\
        Skład podgrupy:
        \begin{enumerate}
            \item Grzegorz Wyborski 260906
            \item Dawid Trzeciak
        \end{enumerate}
        Termin zajęć: Pon, 9:15--11:00 \\
        Prowadzący: Dr inż. Michał Ostrycharczyk\\
        Data wykonania ćwiczenia: #4.2022 r.\\ 
        Data oddania sprawozdania: #5.2022 r.\\
        \vspace{20pt}
        \underline{Sprawozdanie powinno zawierać:}\\
        \begin{enumerate}
            \itemsep 2pt
            \item Podstawy teoretyczne
            \item  Schemat układu pomiarowego
            \item Wykaz przyrządów pomiarowych
            \item Tabele pomiarowo-wynikowe
            \item Przykłady obliczeń
            \item Wykresy podanych zależności
            \item Uwagi, spostrzeżenia i wnioski
            \item Podpisany protokół z badań
        \end{enumerate}
    \end{flushleft}
    \pagebreak
}

\newcommand{\titlewodorowe}[5]{
    \begin{center}
        \fontsize{18}{1}\selectfont{ Wydział Mechaniczno-Energetyczny\\
            Politechnika Wrocławska }\\ \vspace{10pt}
        \textbf{\fontsize{26}{30}\selectfont TECHNOLOGIE WODOROWE\\ LABORATORIUM\\}
        \fontsize{16}{30}\selectfont Sprawozdanie nr. #1\\
        \textbf{Temat:}#2\\
    \end{center}
    \begin{flushleft}
        Grupa nr.#3 \\
        Skład podgrupy:
        \begin{enumerate}
            \item Grzegorz Wyborski 260906
            \item ---
        \end{enumerate}
        Termin zajęć: Pon, 13:15--15:00 \\
        Prowadzący: Dr inż. Mateusz Wnukowski,Dr inż. Monika Tkaczuk\\
        Data wykonania ćwiczenia: #4.2022 r.\\ 
        Data oddania sprawozdania: #5.2022 r.\\
        \vspace{20pt}
        \underline{Sprawozdanie powinno zawierać:}\\
        \begin{enumerate}[\itemsep=2pt]
            \item Podstawy teoretyczne
            \item  Schemat układu pomiarowego
            \item Wykaz przyrządów pomiarowych
            \item Tabele pomiarowo-wynikowe
            \item Przykłady obliczeń
            \item Wykresy podanych zależności
            \item Uwagi, spostrzeżenia i wnioski
            \item Podpisany protokół z badań
        \end{enumerate}
    \end{flushleft}
    \pagebreak
}

\newcommand{\titlePUKE}[1]{
    \begin{flushleft}
        \includegraphics{Logo_PWR.png}\\

        \fontsize{11}{1}\selectfont{
            \vspace{30pt}
            \textbf{Wydział Mechaniczno-Energetyczny}\\
            Kierunek studiów: \textbf{Odnawialne Źródła Energii}\\
            Specjalność: \textbf{Przemysłowe Instalacje OZE}\\
            }
            
            \vspace{10pt}
    \end{flushleft}

    
    \begin{center}
        \textbf{
            \fontsize{20}{30}\selectfont 
            PODSTAWY KONSTRUKCJI URZĄDZEŃ ENERGETYCZNYCH\\}
            
        \vspace{20pt}

        \textbf{
            \fontsize{18}{20}\selectfont 
            #1
            % Rurowy wymiennik ciepła typu: \\
            % Rura w rurze - rury gięte\\
            }
            
        \vspace{50pt}
        \fontsize{16}{20}\selectfont Grzegorz Wyborski\\
        \vspace{50pt}
        Prowadzący:\\
        Dr. inż. Beata Anwajler\\
        \vfill
        Wrocław 2022

        \vspace{20pt}
        % #1
    \end{center}
    \pagebreak
}
\newcommand{\mmTOm}[1]{
    \ensuremath{\frac{#1mm}{1000}}
}

\newcommand{\gTOkg}[1]{
    \ensuremath{\frac{#1g}{1000}}
}

\newcommand{\AREAm}{
    \ensuremath{m^2}
}

\newcommand{\AREAmm}{
    \ensuremath{mm^2}
}

\newcommand{\VOLUMEm}{
    \ensuremath{m^3}
}

\newcommand{\VOLUMEdm}{
    \ensuremath{dm^3}
}

\newcommand{\VELOCITY}{
    \ensuremath{\frac{m}{s}}
}

\newcommand{\ACCELERATION}{
    \ensuremath{\frac{m}{s^2}}
}

\newcommand{\FLOWRATEs}{
    \ensuremath{\frac{m^3}{s}}
}

\newcommand{\FLOWRATEh}{
    \ensuremath{\frac{m^3}{h}}
}

\newcommand{\FLOWRATEkgs}{
    \ensuremath{\frac{kg}{s}}
}

\newcommand{\CONVECTIONa}{
    \ensuremath{\frac{W}{m^2 \cdot K}}
}

\newcommand{\COMDUCTIONl}{
    \ensuremath{\frac{W}{m \cdot K}}
}

\newcommand{\DEGc}{
    \ensuremath{^\circ C}
}

\newcommand{\VISCOSITYkinematic}{
    \ensuremath{\frac{m^2}{s}}
}

\newcommand{\VISCOSITYdynamic}{
    \ensuremath{Pa \cdot s}
}

\newcommand{\DENSITY}{
    \ensuremath{\frac{kg}{m^3}}
}

\newcommand{\SPECIFICHEAT}{
    \ensuremath{\frac{J}{kg \cdot K}}
}









\begin{document}
    \titleterma{18}{temat}{N02-20g}{17.03}{24.03}

    \includepdf{Figures/PDF/Protokol_pomiarowy.pdf}

    \tableofcontents
    \listoffigures
    \listoftables
    \begin{landscape}
        \begin{figure}
            \section{Stanowisko pomiarowe}
            \centering
            \includegraphics[width=\linewidth-142px]
                {Figures/PDF/Stanowisko_pomiarowe.pdf}
                \caption{Shemat budowy stanowiska pomiarowego z wyszczególnionymi elementami oraz pokazanym sposobem podlączenia.}
        \end{figure}
    \end{landscape}
    \thispagestyle{empty}
\newcolumntype{C}{>{\centering\arraybackslash}}
\begin{figure}
    \section{Tabele Wynikowo-pomiarowe}
    \includegraphics[width=\textwidth]{data/fits_table.png}
    \caption{Tabela przedstawiająca ocenę doboru danej pompy.}
\end{figure}

\begin{figure}
    \includegraphics[width=\textwidth]{data/distance_table.png}
    \caption{Tabela przedstawiająca punkty pracy danych pomp oraz jak daleko od założonego punktu pracy się znajdują na wykresie w linii prostej. }
\end{figure}
    \include{Sections/przykładowe_obliczenia.tex}
    \section{Wykresy}
\begin{figure}[h!]
  \centering
  \begin{tikzpicture}
    \begin{axis}[
      width=\textwidth,
      grid = both,
      major grid style = gray,
      xmin=0,xmax=200,
      xtick={0,50,100,...,200},
      ymin=0,
      ymax=200,
      % ytick={0,10,...,100},
       xlabel={Strumień przepływu, \(\FLOWRATEs\)},
       ylabel={Wysokość podnoszenia,\(m\)},
       legend pos=north west,
       legend entries={\(H(Q)\) , \(\delta H_u(Q)\)},
      ]
      \addplot [red,no marks] table [x=Q,y=H] {Figures/data/dataset_NHVE-65-250-1.csv};
      \addplot [blue, no marks] table [x=Q,y=dHu] {Figures/data/dataset_NHVE-65-250-1.csv};
      % \addplot [domain=100:1300,samples=500,color=blue] {64/x};
      \filldraw[thick,black](147,91)circle[radius=2pt]{};
      
    \end{axis}
  \end{tikzpicture}
  \caption{Krzywe pracy dla pompy NHVE-65-250/1. Przecięcie stanowi rzeczywisty punkt pracy, natomiast czarny punkt oznacza punkt pracy dla którego pompa była dobierana. Dla tych danych wejściowych ta pompa jest za mała.}  
\end{figure}


\begin{figure}
  \centering
  \begin{tikzpicture}
    \begin{axis}[
      width=\textwidth,
      grid = both,
      major grid style = gray,
      xmin=0,xmax=200,
      xtick={0,50,100,...,200},
      ymin=0,
      ymax=200,
      % ytick={0,10,...,100},
       xlabel={Strumień przepływu, \(\FLOWRATEs\)},
       ylabel={Wysokość podnoszenia,\(m\)},
       legend pos=north west,
       legend entries={\(H(Q)\) , \(\delta H_u(Q)\)},
      ]
      \addplot [red,no marks] table [x=Q,y=H] {Figures/data/dataset_NHVE-80-250-1.csv};
      \addplot [blue, no marks] table [x=Q,y=dHu] {Figures/data/dataset_NHVE-80-250-1.csv};
      % \addplot [domain=100:1300,samples=500,color=blue] {64/x};
      \filldraw[thick,black](147,91)circle[radius=2pt]{};
      
    \end{axis}
  \end{tikzpicture}
  \caption{Krzywe pracy dla pompy NHVE-80-250/1. Przecięcie stanowi rzeczywisty punkt pracy, natomiast czarny punkt oznacza punkt pracy dla którego pompa była dobierana. Dla tych danych wejściowych ta pompa jest przewymiarowana.}  
\end{figure}


\begin{figure}
  \centering
  \begin{tikzpicture}
    \begin{axis}[
      width=\textwidth,
      grid = both,
      major grid style = gray,
      xmin=0,xmax=200,
      xtick={0,50,100,...,200},
      ymin=0,
      ymax=200,
      % ytick={0,10,...,100},
       xlabel={Strumień przepływu, \(\FLOWRATEs\)},
       ylabel={Wysokość podnoszenia,\(m\)},
       legend pos=north west,
       legend entries={\(H(Q)\) , \(\delta H_u(Q)\)},
      ]
      \addplot [red,no marks] table [x=Q,y=H] {Figures/data/dataset_NHVE-125-400-A.csv};
      \addplot [blue, no marks] table [x=Q,y=dHu] {Figures/data/dataset_NHVE-125-400-A.csv};
      % \addplot [domain=100:1300,samples=500,color=blue] {64/x};
      \filldraw[thick,black](147,91)circle[radius=2pt]{};
      
    \end{axis}
  \end{tikzpicture}
  \caption{Krzywe pracy dla pompy NHVE-125-400/A. Przecięcie stanowi rzeczywisty punkt pracy, natomiast czarny punkt oznacza punkt pracy dla którego pompa była dobierana. Dla tych danych wejściowych ta pompa jest znacząco za mała.}  
\end{figure}
    \section{Wnioski}
\paragraph{}{Dla zadanych wag i parametrów najlepiej pasuje pompa NHVE-65-250/1, co wynika zarówno z wykresu, jak i z oceny doboru.
Problemem może być, że rzeczywisty punkt pracy znajduję się poniżej zakładanego.}
\paragraph{}{Kolejna dobierana pompa to NHVE-80-250/1, jest przewymiarowana, jednak nieznacznie. Należy przeprowadzić procedurę doboru dla pompy o oznaczeniu NHVE-80-250/2, czyli tego samego model, ale o zmniejszonym wirniku.}
\paragraph{}{Ostatnia dobierana pompa to NHVE-125-400/A. Ten model jest znacząco niedowymiarowany i nie nadaje się do danego zastosowania. Już na wykresie przedstawiającym pola pracy było widać, że żadna pompa Pracująca na prędkości obrotowej  \(1500\frac{obr.}{min}\) nie pozwoli uzyskać zakładanych parametrów.}

     
    %\bibliographystyle{plain} 
    %\bibliography{main}
\end{document}