\section{Wykresy}
\begin{figure}[h!]
  \centering
  \begin{tikzpicture}
    \begin{axis}[
      width=\textwidth,
      grid = both,
      major grid style = gray,
      xmin=0,xmax=200,
      xtick={0,50,100,...,200},
      ymin=0,
      ymax=200,
      % ytick={0,10,...,100},
       xlabel={Strumień przepływu, \(\FLOWRATEs\)},
       ylabel={Wysokość podnoszenia,\(m\)},
       legend pos=north west,
       legend entries={\(H(Q)\) , \(\delta H_u(Q)\)},
      ]
      \addplot [red,no marks] table [x=Q,y=H] {Figures/data/dataset_NHVE-65-250-1.csv};
      \addplot [blue, no marks] table [x=Q,y=dHu] {Figures/data/dataset_NHVE-65-250-1.csv};
      % \addplot [domain=100:1300,samples=500,color=blue] {64/x};
      \filldraw[thick,black](147,91)circle[radius=2pt]{};
      
    \end{axis}
  \end{tikzpicture}
  \caption{Krzywe pracy dla pompy NHVE-65-250/1. Przecięcie stanowi rzeczywisty punkt pracy, natomiast czarny punkt oznacza punkt pracy dla którego pompa była dobierana. Dla tych danych wejściowych ta pompa jest za mała.}  
\end{figure}


\begin{figure}
  \centering
  \begin{tikzpicture}
    \begin{axis}[
      width=\textwidth,
      grid = both,
      major grid style = gray,
      xmin=0,xmax=200,
      xtick={0,50,100,...,200},
      ymin=0,
      ymax=200,
      % ytick={0,10,...,100},
       xlabel={Strumień przepływu, \(\FLOWRATEs\)},
       ylabel={Wysokość podnoszenia,\(m\)},
       legend pos=north west,
       legend entries={\(H(Q)\) , \(\delta H_u(Q)\)},
      ]
      \addplot [red,no marks] table [x=Q,y=H] {Figures/data/dataset_NHVE-80-250-1.csv};
      \addplot [blue, no marks] table [x=Q,y=dHu] {Figures/data/dataset_NHVE-80-250-1.csv};
      % \addplot [domain=100:1300,samples=500,color=blue] {64/x};
      \filldraw[thick,black](147,91)circle[radius=2pt]{};
      
    \end{axis}
  \end{tikzpicture}
  \caption{Krzywe pracy dla pompy NHVE-80-250/1. Przecięcie stanowi rzeczywisty punkt pracy, natomiast czarny punkt oznacza punkt pracy dla którego pompa była dobierana. Dla tych danych wejściowych ta pompa jest przewymiarowana.}  
\end{figure}


\begin{figure}
  \centering
  \begin{tikzpicture}
    \begin{axis}[
      width=\textwidth,
      grid = both,
      major grid style = gray,
      xmin=0,xmax=200,
      xtick={0,50,100,...,200},
      ymin=0,
      ymax=200,
      % ytick={0,10,...,100},
       xlabel={Strumień przepływu, \(\FLOWRATEs\)},
       ylabel={Wysokość podnoszenia,\(m\)},
       legend pos=north west,
       legend entries={\(H(Q)\) , \(\delta H_u(Q)\)},
      ]
      \addplot [red,no marks] table [x=Q,y=H] {Figures/data/dataset_NHVE-125-400-A.csv};
      \addplot [blue, no marks] table [x=Q,y=dHu] {Figures/data/dataset_NHVE-125-400-A.csv};
      % \addplot [domain=100:1300,samples=500,color=blue] {64/x};
      \filldraw[thick,black](147,91)circle[radius=2pt]{};
      
    \end{axis}
  \end{tikzpicture}
  \caption{Krzywe pracy dla pompy NHVE-125-400/A. Przecięcie stanowi rzeczywisty punkt pracy, natomiast czarny punkt oznacza punkt pracy dla którego pompa była dobierana. Dla tych danych wejściowych ta pompa jest znacząco za mała.}  
\end{figure}